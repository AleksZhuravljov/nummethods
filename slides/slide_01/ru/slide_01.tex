\documentclass[12pt]{beamer}
\usepackage[T2A]{fontenc}
\usetheme{Warsaw}
\setbeamertemplate{page number in head/foot}[totalframenumber]
\usepackage[utf8]{inputenc}
\usepackage[russian, english]{babel}
\usepackage{amsmath}
\usepackage{amsfonts}
\usepackage{amssymb}
\usepackage{graphicx}
\author{Александр Сергеевич Журавлёв}
\title[Дифференциальные уравнения]{Численные методы \\ Дифференциальные уравнения}
\setbeamercovered{transparent} 
%\setbeamertemplate{navigation symbols}{} 
%\logo{} 
\institute{Физико-технический институт} 
%\date{} 
%\subject{} 
\begin{document}

\begin{frame}
    \titlepage
\end{frame}

%\begin{frame}
%\tableofcontents
%\end{frame}

\begin{frame}{Темы}
    \begin{itemize}
        \item Исчисление бесконечно малых
            \begin{itemize}
                \item Геометрическая интерпретация
                \item Дифференциал и полная производная                
                \item Частная производная
            \end{itemize}
        \item Метод конечных разностей
            \begin{itemize}
                \item Разложение в ряд Тейлора        
                \item $\frac{\partial}{\partial t} \rho c_{\rho} T = \vec{\nabla}  \lambda \vec{\nabla} T$      
            \end{itemize}
        \item Метод конечного объёма
            \begin{itemize}
                \item $\int \limits_{V} \frac{\partial}{\partial t} \rho c_{\rho} T dV = \oint \limits_{S}  \lambda \vec{\nabla} T d \vec{S}$               
            \end{itemize}
        \item Метод конечных элементов          
    \end{itemize}
\end{frame}

\begin{frame}{Источники}
\end{frame}

\begin{frame}{Разложение в ряд Тейлора}
\begin{eqnarray}
AU=0,\; LU=0,\; AU_{i}=LU_{i}+R_{i},\\
x_{0}, x_{1}, \dots, x_{N+1},\\
x_{0}=0, x_{N+1}=L,\; h=x_{i+1}-x_{i}=\frac{L}{N+1},\\
U_{i+1}=U_{i}+U'_{i}h+U''_{i}\frac{h^2}{2}+U'''_{i}\frac{h^3}{6}+U^{IV}_{i}\frac{h^4}{24}+\cdots\\
U_{i-1}=U_{i}-U'_{i}h+U''_{i}\frac{h^2}{2}-U'''_{i}\frac{h^3}{6}+U^{IV}_{i}\frac{h^4}{24}-\cdots
\end{eqnarray}
\end{frame}

\begin{frame}{Разложение в ряд Тейлора}
\begin{eqnarray}
U'{i}=\frac{U_{i+1}-U_{i}}{h}+R^f_{i},\; R^f_{i}=-U''_{i}\frac{h}{2}-U'''_{i}\frac{h^2}{6}-\cdots\\
U'{i}=\frac{U_{i}-U_{i-1}}{h}+R^b_{i},\; R^b_{i}=U''_{i}\frac{h}{2}-U'''_{i}\frac{h^2}{6}-\cdots\\
U'{i}=\frac{U_{i+1}-U_{i-1}}{2h}+R^c_{i},\; R^c_{i}=-U'''_{i}\frac{h^2}{6}-\cdots\\
U'{i}=\frac{U_{i+1}-2U{i}+U_{i-1}}{h^2}+R^2_{i},\; R^2_{i}=-U^{IV}_{i}\frac{h^2}{12}-\cdots
\end{eqnarray}
\end{frame}

\end{document}