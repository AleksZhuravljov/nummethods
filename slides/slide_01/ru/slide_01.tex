\documentclass[12pt]{beamer}
\usepackage[T2A]{fontenc}
\usetheme{Warsaw}
\setbeamertemplate{page number in head/foot}[totalframenumber]
\usepackage[utf8]{inputenc}
\usepackage[russian, english]{babel}
\usepackage{amsmath}
\usepackage{amsfonts}
\usepackage{amssymb}
\usepackage{graphicx}
\author{Александр Сергеевич Журавлёв}
\title[Дифференциальные уравнения]{Численные методы \\ Дифференциальные уравнения}
\setbeamercovered{transparent} 
%\setbeamertemplate{navigation symbols}{} 
%\logo{} 
\institute{Физико-технический институт} 
%\date{} 
%\subject{} 
\begin{document}

\begin{frame}
    \titlepage
\end{frame}

%\begin{frame}
%\tableofcontents
%\end{frame}

\begin{frame}{Темы}
    \begin{itemize}
        \item Исчисление бесконечно малых
            \begin{itemize}
                \item Геометрическая интерпретация
                \item Дифференциал, полная и частная производные                
                \item Функционал и его вариация
                \item Принцип наименьшего действия
            \end{itemize}
        \vspace{0.2 cm}    
        \item Вычислительная линейная алгебра
            \begin{itemize}
                \item Линеаризация
                \item Разряженные матрицы
                \item Алгоритмирование матричных операций
            \end{itemize}                                          
    \end{itemize}
\end{frame}

\begin{frame}{Темы}
    \begin{itemize}
    \item Метод конечных разностей (МКР)
            \begin{itemize}
                \item Разложение в ряд Тейлора        
                \item $\rho \frac{\partial}{\partial t} c_{\rho} T + \rho \vec{v} \vec{\nabla} c_{\rho} T = \vec{\nabla}  \lambda \vec{\nabla} T$
                \item Сеточные числа Фурье и Куранта
                \item Общий алгоритм численного решения      
            \end{itemize}     
            \vspace{0.2 cm}  
        \item Метод конечных объёмов (МКО)
            \begin{itemize}
                \item Разбиение на элементарные объемы
                \item Интерполяция между центрами объемов
                \item $\int \limits_{V} \frac{\partial}{\partial t} \rho c_{\rho} T dV + \oint \limits_{S}  \rho c_{\rho} \vec{v}  T d \vec{S}= \oint \limits_{S}  \lambda \vec{\nabla} T d \vec{S}$               
            \end{itemize}                   
    \end{itemize}
\end{frame}

\begin{frame}{Темы}
    \begin{itemize}    
        \item Метод конечных элементов (МКЭ)       
            \begin{itemize}
                \item Вариационный принцип \\      
                $\Pi\left(T\right) = \int \limits_{V} \left( 
                \frac{1}{2}a \left(\vec{\nabla}T\right)^2 - QT\right)dV +
                \oint \limits_{S}  \vec{q}  T d \vec{S}$\\
               $ \delta \Pi = - \int \limits_{V} \left( 
                \vec{\nabla}a\vec{\nabla}T - Q\right) \delta T dV +
                \oint \limits_{S} \left(a\vec{\nabla}T + \vec{q}  \right) \delta T d \vec{S}$
                \item Конечные элементы и координатные функции
                \item Триангуляция Делоне 
                \item Минимизации функционала   
                \item Метод Галёркина                         
            \end{itemize}              
    \end{itemize}
\end{frame}

\begin{frame}{Источники}
    \begin{itemize}
        \item Самарский, А.А., 1978. Методы решения сеточных уравнений. Наука.
        \item Самарский, А.А. and Гулин, А.В., 2003. Численные методы математической физики. М: Науч. мир.
        \item Зенкевич, О., 1975. Метод конечных элементов в технике. Рипол Классик.
        \item LeVeque, R.J., 2007. Finite difference methods for ordinary and partial differential equations: steady-state and time-dependent problems (Vol. 98). Siam.
        \item Eymard, R., Gallouët, T. and Herbin, R., 2000. Finite volume methods. Handbook of numerical analysis, 7, pp.713-1018.
    \end{itemize}
\end{frame}

\begin{frame}{Разложение в ряд Тейлора}
    \begin{eqnarray}
        AU=0,\; LU=0,\; AU_{i}=LU_{i}+R_{i},\\
        x_{0}, x_{1}, \dots, x_N,\\
        x_{0}=0, x_N=L,\;\; h=x_{i+1}-x_{i},\\
        f\left(x, a\right) = \sum\limits_{n=0}^{\infty} \frac{f^{\left(n\right)}\left(a\right)}{n!} \left(x - a\right)^n,\\
        U_{i+1}=U_{i}+U'_{i}h+U''_{i}\frac{h^2}{2}+U'''_{i}\frac{h^3}{6}+U^{IV}_{i}\frac{h^4}{24}+\cdots,\\
        U_{i-1}=U_{i}-U'_{i}h+U''_{i}\frac{h^2}{2}-U'''_{i}\frac{h^3}{6}+U^{IV}_{i}\frac{h^4}{24}-\cdots.
    \end{eqnarray}
\end{frame}

\begin{frame}{Разложение в ряд Тейлора}
    \begin{eqnarray}
        U'_{i}=\frac{U_{i+1}-U_{i}}{h}+R^f_{i},\;\; R^f_{i}=-U''_{i}\frac{h}{2}-U'''_{i}\frac{h^2}{6}-\cdots,\\
        U'_{i}=\frac{U_{i}-U_{i-1}}{h}+R^b_{i},\;\; R^b_{i}=U''_{i}\frac{h}{2}-U'''_{i}\frac{h^2}{6}-\cdots,\\
        U'_{i}=\frac{U_{i+1}-U_{i-1}}{2h}+R^c_{i},\;\; R^c_{i}=-U'''_{i}\frac{h^2}{6}-\cdots,\\
        U''_{i}=\frac{U_{i+1}-2U{i}+U_{i-1}}{h^2}+R^2_{i},\;\; R^2_{i}=-U^{IV}_{i}\frac{h^2}{12}-\cdots.
    \end{eqnarray}
\end{frame}

\begin{frame}{МКР Уравнение теплопроводности}
    \begin{eqnarray}
        \frac{\partial T}{\partial t} + v \frac{\partial T}{\partial x}  =  \frac{\lambda}{\rho c_{\rho}} \frac{\partial^2 T}{\partial x^2},\\
        \frac{T_i^{n+1} - T_i^{n}}{\Delta t} + v_{i}\frac{T_{i+1}^{n}-T_{i-1}^{n}}{2\Delta x} =  \frac{\lambda}{\rho c_{\rho}}\frac{T_{i-1}^{n} - 2T_i^{n} + T_{i+1}^{n}}{\Delta x^2},\\
        \frac{T_i^{n+1} - T_i^{n}}{\Delta t} + v_{i}\frac{T_{i+1}^{n+1}-T_{i-1}^{n+1}}{2\Delta x} =  \frac{\lambda}{\rho c_{\rho}}\frac{T_{i-1}^{n+1} - 2T_i^{n+1} + T_{i+1}^{n+1}}{\Delta x^2},\\       
        T_i^{n+1} = \left(F + \frac{C}{2} \right) T_{i-1}^{n} + \left(1 - 2F \right) T_i^{n} + \left(F - \frac{C}{2} \right) T_{i+1}^{n},\\
        \left(-\frac{C}{2} - F \right) T_{i-1}^{n+1} + \left(1 + 2F \right) T_i^{n+1} + \left(\frac{C}{2} - F \right) T_{i+1}^{n+1} = T_i^{n}.
    \end{eqnarray}
\end{frame}

\begin{frame}{МКР Сеточные числа Фурье и Куранта}
    \begin{eqnarray}
        F = \frac{\lambda}{\rho c_{\rho}} \frac{\Delta t}{\Delta x^2}, \; \; C = v_{i} \frac{\Delta t}{\Delta x},\\
        F < \frac{1}{2}, \; \; C < 2F,\\
        \Delta t < \frac{\Delta x^2}{2} \frac{\rho c_{\rho}}{\lambda}, \; \; \Delta x < \frac{2}{v_{i}} \frac{\lambda}{\rho c_{\rho}},
    \end{eqnarray}
\end{frame}

\begin{frame}{МКР Уравнение теплопроводности}
 \begin{eqnarray}
    \Pi\left(T\right) = \int \limits_{L} \left(\frac{1}{2}a \left(\frac{d T}{d x}\right)^2 - QT\right)dx +
        \left.  q  T \right|_\Gamma,\\
        \delta \Pi = - \int \limits_{L} \left(\frac{d}{dx}a\frac{dT}{dx} - Q\right) \delta T dL +
    \left.  \left(a\frac{dT}{dx} + q  T \right) \delta T\right|_\Gamma,\\
    T\left(x\right) = \sum\limits_{j=1}^{N-1}\sum\limits_{i=1}^N \alpha_i f^j_i\left(x\right)=
       \sum\limits_{j=1}^{N-1} \sum\limits_{i=1}^N T_i f^j_i\left(x\right),\\    
    \Pi = \sum\limits_{j=1}^{N-1} \Pi^j \Rightarrow  \frac{\partial}{\partial T_i} \sum\limits_{j=1}^{N-1} \Pi^j= 0,
    \end{eqnarray}
\end{frame}

\end{document}