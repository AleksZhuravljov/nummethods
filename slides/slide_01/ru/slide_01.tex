\documentclass[12pt]{beamer}
\usepackage[T2A]{fontenc}
\usetheme{Warsaw}
\setbeamertemplate{page number in head/foot}[totalframenumber]
\usepackage[utf8]{inputenc}
\usepackage[russian, english]{babel}
\usepackage{amsmath}
\usepackage{amsfonts}
\usepackage{amssymb}
\usepackage{graphicx}
\author{Александр Сергеевич Журавлёв}
\title[Дифференциальные уравнения]{Численные методы \\ Дифференциальные уравнения}
\setbeamercovered{transparent} 
%\setbeamertemplate{navigation symbols}{} 
%\logo{} 
\institute{Физико-технический институт} 
%\date{} 
%\subject{} 
\begin{document}

\begin{frame}
    \titlepage
\end{frame}

%\begin{frame}
%\tableofcontents
%\end{frame}

\begin{frame}{Темы}
    \begin{itemize}
        \item Исчисление бесконечно малых
            \begin{itemize}
                \item Геометрическая интерпретация
                \item Дифференциал и полная производная                
                \item Частная производная
            \end{itemize}
        \item Метод конечных разностей
            \begin{itemize}
                \item Разложение в ряд Тейлора        
                \item $\frac{\partial}{\partial t} \rho c_{\rho} T = \vec{\nabla}  \lambda \vec{\nabla} T$      
            \end{itemize}
        \item Метод конечного объёма
            \begin{itemize}
                \item $\int \limits_{V} \frac{\partial}{\partial t} \rho c_{\rho} T dV = \oint \limits_{S}  \lambda \vec{\nabla} T d \vec{S}$               
            \end{itemize}
        \item Метод конечных элементов          
    \end{itemize}
\end{frame}

\begin{frame}{Источники}
\end{frame}

\end{document}